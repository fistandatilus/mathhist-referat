% !TeX spellcheck = ru_RU
\documentclass[a4paper,12pt]{extarticle}
\usepackage[T2A]{fontenc}
\usepackage[utf8]{inputenc}
\usepackage[english,russian]{babel}
\usepackage{amsmath}
\usepackage{amssymb}
\usepackage{amsthm}
\usepackage{listings}
\usepackage{hyperref}

\usepackage{xcolor}

\usepackage{indentfirst}
\usepackage[left=25mm, top=35mm, right=25mm, bottom=25mm, nohead, footskip=5mm]{geometry}
\renewcommand{\baselinestretch}{1.5}
\usepackage[indent=1.25cm]{parskip}

\newtheorem{theorem}{Теорема}[subsection]
\renewcommand{\thetheorem}{\arabic{subsection}.\arabic{theorem}}
\newtheorem{utv}{Утверждение}
\theoremstyle{definition}
\newtheorem{definition}{Определение}

\newcommand{\cit}[1]{<<\textit{#1}>>}
\newcommand{\car}[1]{\overline{\overline{#1}}}

\author{Борисенков Никита}



\begin{document}

\begin{titlepage}
    \newpage
    \begin{center}
        Московский государственный университет имени M.В. Ломоносова
        Механико-математический факультет\\
    \end{center}

    \vspace{8em}

    \begin{center}
        \Large Реферат\\
    \end{center}

    \vspace{2em}

    \begin{center}
        \textsc{\textbf{Г. Кантор. К обоснованию учения о трансфинитных множествах}}
    \end{center}

    \vspace{20em}



    \newbox{\lbox}
    \newlength{\maxl}
    \setlength{\maxl}{\wd\lbox}
    \hfill\parbox{13cm}{
        \hspace*{5cm}\hspace*{-5cm}Студент: \qquad\qquad\hbox to \maxl{Борисенков Никита Николаевич\hfill}\\
        \\
        \hspace*{5cm}\hspace*{-5cm}Группа:\qquad\qquad $\;\:$ \hbox to\maxl{410}\\
    }


    \vspace{\fill}

    \begin{center}
        Москва \\2025
    \end{center}

\end{titlepage}
\newpage
\tableofcontents
\newpage
\section{Введение}
В данном реферате обозревается работа Георга Кантора <<К обоснованию учения о трансфинитных множествах>>\cite{orig}.
Текст взят в русском переводе Ф. А. Медведева\cite{used}.
Эта работа является последней из работ Кантора по теории множеств и здесь теория множеств сформулирована в наиболее завершённом виде.
В настоящем реферате рассмотрены параграфы 1 -- 10, в которых Кантор формулирует основания теории множеств, теории порядковых типов.
Я также привожу краткую биография Георга Кантора.

\section{Биография Кантора}
Информация для составления данной биографии бралась из \cite{bio}.

Георг Кантор родился в Санкт-Петербурге 3 марта 1845 года в семье Георга Вольдемара Кантора и Марии Кантор, урождённой Бём.
Его отец работал маклёром на Бирже, был родом из Копенгагена.
Мать родилась в Петербурге в семье скрипачей, сама играла на скрипке и учила детей игре на скрипке и фортепиано.
Кантор сохранил любовь к скрипке на всю жизнь.
Всего в семье было 4 детей, Георг Кантор был старшим.
Жили они в доме купца Траншеля на 11-й линии Васильевского острова, позже они переехали в дом Жадимировского на Большой Конюшенной.
В этом же доме в 1850 году поселился Пафнутий Львович Чебышёв.
Он мог видеть Кантора.

Георга и его брата отдали учиться в Петришуле, Главное немецкое училище при лютеранской церкви Св. Петра.
В один из годов Кантор в течении семестра получал практически одни двойки, хотя в общем был прилежным учеником.
Предположительно это был период депрессии, подобный тем, от которых он страдал позднее в своей жизни.
Класс Кантора насчитывал 67 учеников.

В 1856 году семья из-за усугубившейся чахотки отца переехала в Германию, в Франкфурт-на-Майне.
Переезд планировался временным, судя по тому, что Георг Вольдемар взял отпуск на год, но в Россию они не вернулись.

Отец хотел, чтобы Георг стал инженером, но он выбрал математику, для изучения которой поступил в Политехникум в Цюрихе.
Георг Вольдемар Кантор умер в 1863 году в Гейдельберге.
После смерти отца Кантор стал учиться в Берлинском университете и окончил его в 1867 году.
Учителями Кантора были Куммер, Кронекер и Вейерштрасс.
Последний оказал на него особое влияние.
В 1867 году в Берлине Кантор под руководством Куммера защитил диссертацию по теории чисел
<<О неопределённых уравнениях второй степени>> (De aequationibus secundi gradus indeterminatis), за которую ему присудили учёную степень.

В 1869 году Гейне, пригласил Кантора работать в университет Галле. Кантор получил звание приват-доцента университета Галле и стал преподавателем математического семинара факультета философии.
В 1872 году Кантор опубликовал работу <<Обобщение одной теоремы из теории тригонометрических рядов>>, в которой Кантор определяет действительные числа как пределы последовательностей чисел.
В том же году Дедекинд сформулировал свой метод сечений.
Кантор и Дедекинд с апреля 1872 года вели переписку, в которой они обсуждали непрерывность и иррациональные числа.

Кантор ещё до формулирования основных понятий теории множеств начинает рассматривать взаимно-однозначное соответствие числовых последовательностей как основное понятие.
В 1874 году Кантор публикует работу <<Об одном свойстве совокупности всех действительных алгебраических чисел>>, в которой он показывает, что алгебраические числа взаимно однозначно отображаются на целые положительные.
В том же году он женился на Валли Марии Софи Гутман, детской подруге его сестры Софии.
Во время медового месяца им сопутствовал Дедекинд, в математических беседах с которым Кантор проводил много времени.
У Канторов родилось шестеро детей.

В 1878 году вышла первая крупная работа Кантора <<К учению о многообразиях>> (Ein Beitrag zur Mannigfaltiggkeitslehre), в которой он вводит понятие эквивалентности или равномощности,
доказывая существование взаимно-однозначного соответствия между одномерными и многомерными непрерывными объектами.
Здесь же впервые формулируется континуум-гипотеза.
В 1879 году Кантор избран членом-корреспондентом Общества наук в Геттингене.
Вышла его работа <<Об одной теореме из теории непрерывных многообразий>> (Über einen Satz ausder Theorie der stetigen Mannigfaltigkeiten),
в которой он продолжает устанавливать соответствие между объектами различной размерности.
С 1879 по 1884 год Кантор печатает цикл статей под названием <<О бесконечных линейных точечных многообразиях>> (Über unnennliche lineare Punktmannigfaltigkeit).

Кантор мечтал работать в Берлинском университете, но этому сопротивлялся заведующий кафедрой математики Кронекер.
Кронекер был основателем конструктивной математики, а теория множеств исследовала свойства множеств без их конкретного представления.
Кантор тяжело переживал его критику.
После неудачной попытки стать профессором в Берлине у Кантора начинается депрессивный эпизод.
С 1884 года Кантор вынужден лечиться в психиатрической клинике.
С этого времени его интересы смещаются в область филосовского обоснования теории множеств.

В 1889 году Кантор принят в Немецкую академию естественных наук Леопольдина.

Постепенно растёт авторитет Кантора среди математиков.
В 1890 году Кантор выступает инициатором создания Немецкого математического общества Deutsche Mathematiker-Vereinigung.
Растёт популярность теории множеств и её признание среди европейских математиков.

В 1890/1891 году вышла работа Кантора <<Об одном элементарном вопросе учения о многообразиях>> с изложением диагонального метода.

В 1895–1897 годах он написал последнюю, незавершённую работу <<К обоснованию учения о трансфинитных множествах>> (Beiträge zur Begründung der transfiniten Mengenlehr) по теории множеств.
В ней его теория представлена в наиболее оформленном виде. 

В октябре 1896 года умерла Мария Кантор, мать Георга, в декабре 1899 года умер его сын Константин.
Эти события негативно влияли на психическое состояние Кантора и усугубляли его болезнь.

Кантор был инициатором и участником Первого международного конгресса математиков в Цюрихе в 1897 году, делал доклад о парадоксах теории множеств на втором заседании Математического общества в 1903 году,
принимал участие в Международном математическом конгрессе в Гейдельберге в 1904 году.
На последнем прозвучала критика Кёнига теорий Кантора, которую Георг тяжело переживал.

В 1913 году Кантор ушёл в отставку. Началась Первая мировая война. Кантор страдал от бедности и недоедания.
В конце жизни он много раз попадал в психическую клинику.
Последний раз Кантор попал в клинику в 1917, где умер от сердечного приступа 6 января 1918 года.

\section{Основная часть}
\subsection{\S 1. Понятие мощности или кардинального числа}
В первом параграфе Кантор вводит понятия множества, объединения множеств, а также приводит базовые определения и факты о кардинальности.

Он не даёт чёткого определения множества, но говорит, что это \cit{соединение в некое целое $M$ определённых хорошо различимых предметов $m$ нашего созерцания или нашего мышления.}
Объединение непересекающихся множеств, которые мы сейчас обозначаем как $M \sqcup N$, Кантор обозначает так: $(M, N)$.
Обозначения для подмножества он не даёт, но даёт следующее простое утверждение о них:
\cit{Если $M_2$ является частью $M_1$, а $M_1$ -- частью $M$, то и $M_2$ является частью $M$.}

Далее Кантор вводит понятие мощности или, иначе, кардинального числа, которое он определяет в том же стиле, в каком он определял множества.
Конкретно его определение выглядит так:
\cit{<<Мощностью>> или <<кардинальным числом>> множества $М$ мы называем то общее понятие,
которое получается при помощи нашей активной мыслительной способности из $M$,
когда мы абстрагируемся от качества его различных элементов $m$ и от порядка их задания.}
Это определение можно понять как то, что мощность -- это особое множество единиц.
Мощность множества он обозначает так: $\car{M}$.

Сразу после определения мощности Кантор определяет эквивалентность множеств, и в этот раз определение совпадает с современным.
\begin{definition}
Множества $M$ и $N$ называются эквивалентными, если между ними существует взаимно-однозначное соответствие. Обозначается так: $M \sim N$ или $N \sim M$.
\end{definition}

Далее Кантор замечает, что эквивалентность множеств может задаваться отображением, которое будет заданный элемент одного множества переводить в заданный элемент второго множества.
После этого он утверждает, что для отношения эквивалентности множеств верны симметричность и транзитивность.
Далее Кантор формулирует и доказывает следующее утверждение:
\begin{utv}
\label{simeq}
    $M \sim N \iff \car{M} = \car{N}$.
\end{utv}
Завершается параграф следующим утверждением:
\begin{utv}
\label{manysim}
Если множества $M_i$ попарно не пересекаются, множества $M_i'$ тоже попарно не пересекаются и $M_i \sim M_i'$, то тогда $\bigcup\limits_{i \in I} M_i \sim \bigcup\limits_{i \in I} M_i'$
\end{utv}

\subsection{\S 2. <<Больше>> и <<меньше>> для мощностей}
Во втором параграфе Кантор вводит отношения больше и меньше на кардинальных числах.

\begin{definition}
    \label{def-less}
    Кардинальные числа $\mathfrak{a} = \car{M}, \mathfrak{b} = \car{N}$ состоят в отношении $\mathfrak{a} < \mathfrak{b}$, если верны 2 условия:
    \begin{enumerate}
        \item $\forall M_1 \subset M, M_1 \nsim N$
        \item $\exists N_1 \subset N, N_1 \sim M$
    \end{enumerate}
\end{definition}
Далее замечается, что из этого определения следует, что если верно $\mathfrak{a} < \mathfrak{b}$, то не верно ни $\mathfrak{a} = \mathfrak{b}$, ни $\mathfrak{b} < \mathfrak{a}$,
а также то, что для него верна транзитивность.
Отмечаются следующие соотношения для подмножестве:
$$
P_1 \subset P, \mathfrak{a} < \car{P_1} \Rightarrow \mathfrak{a} < \car{P}
$$
$$
P_1 \subset P, \car{P} < \mathfrak{b} \Rightarrow \car{P_1} < \mathfrak{b}
$$

При этом Кантор указывает, что в данный момент рассуждений невозможно показать, что любые 2 кардинальных числа состоят в отношении <, > или =,
и приводит несколько утверждений, которые из этого следуют и будут обсуждены позднее.

\subsection{\S 3. Сложение и умножение мощностей}
В третьем параграфе Кантор вводит сложение и умножение мощностей, а также операции над множествами, которые к ним приводят.

Объединение множеств, которое он ввёл в первом параграфе, он называет суммой.
Сумму же кардинальных чисел Кантор вводит следующим образом: $\mathfrak{a} = \car{M}, \mathfrak{b} = \car{N} \Rightarrow \mathfrak{a + b} = \car{M \sqcup N}$.
Корректность этого определения следует из утверждений \ref{simeq} и \ref{manysim}.
Из того, что мощность не зависит от порядка элементов, получаются коммутативность и ассоциативность сложения.

После этого вводится умножение множеств, которое сейчас известно как декартово произведение.
Кантор даёт следующее определение: \cit{Каждый элемент $m$ множества M можно объединить с каждым элементом $n$ множества N в новый элемент $(m, n)$;
для множества всех этих объединений $(m, n)$ мы примем обозначение $(M \cdot N)$. Назовём его <<\textnormal{Произведение множеств $M$ и $N$}>>.}
Далее показывается, что если $M \sim M'$ и $N \sim N'$, то $(M \cdot N) \sim (M' \cdot N')$.
Из этого получается, что можно корректно определить умножение кардинальных чисел: $\mathfrak{a \cdot b} = \car{(M \cdot N)}$, если $\mathfrak{a} = \car{M}, \mathfrak{b} = \car{N}$.
Также Кантор показывает, что для умножения верна коммутативность, ассоциативность и есть дистрибутивность относительно сложения.

\subsection{\S 4. Возведение мощностей в степень}
В четвёртом параграфе определяется возведение мощностей в степень.

Кантор делает это через множество всех отображений $M \rightarrow N$.
Он называет такие отображения \cit{покрытие $N$ посредством $M$}, а всё множество обозначает $(N|M)$.
Возведение в степень кардинальных чисел определяется так: $\mathfrak{a^b} = \car{(N|M)}$.
Далее сообщается, что для возведения в степень верны следующие соотношения:
$$\mathfrak{a^b \cdot a^c = a^{b+c}}$$
$$\mathfrak{a^c \cdot b^c = (a \cdot b)^c}$$
$$\mathfrak{(a^b)^c = a^{b \cdot c}}$$

В примечании к этому параграфу отмечена важность этих формул и приведён пример их использования для доказательства того,
что мощность континуума в счётной степени совпадает с мощностью континуума.

\subsection{\S 5. Конечные кардинальные числа}
В пятом параграфе рассматривается применение общего учения о кардинальных числах к конечным множествам.

Сначала Кантор строит натуральные числа на основе мощностей конечных множеств.
Множеству из одного элемента $E_0 = \{e_0\}$ сопоставляется единица $1 = \car{E_0}$, далее $E_1 = E_0 \sqcup {e_1}$, $\car{E_1} = 2$ и так далее.
В общем случае $\nu = \car{E_{\nu - 1}}$, $E_{\nu} = E_{\nu} \sqcup \{e_{\nu}\} = \{e_0, ..., e_{\nu}\}$.
Из определения суммы мощностей следует $\car{E_{\nu}} = \car{E_{\nu-1}} + 1$.

Далее Кантор формулирует 3 теоремы о свойствах только что определённых чисел.
Текст теорем перенесён из самой работы.
\begin{theorem}
\label{A}
Все члены неограниченной последовательности конечных кардинальных чисел $$1, 2, 3, ..., \nu,...$$ отличны друг от друга.
\end{theorem}
\begin{theorem}
\label{B}
Каждое из этих чисел $\nu$ больше, чем предшествующее ему, и меньше, чем следующее за ним.
\end{theorem}
\begin{theorem}
\label{C}
Не существует кардинального числа, которое по своей величине было бы расположено между двумя соседними числами $\nu$ и $\nu + 1$.
\end{theorem}
Доказательства этих теорем основываются на двух других теоремах:
\begin{theorem}
\label{D}
Если $M$ -- множество, обладающее свойством не иметь одинаковой мощности ни с каким своим подмножеством, то и множество $(M, e)$,
получаемое из $M$ присоединением  одного нового элемента $e$, обладает тем же свойством не иметь одинаковой мощности ни с каким из его подмножеств.
\end{theorem}
\begin{theorem}
\label{E}
Если $N$ -- множество с конечным кардинальным числом $\nu$, а $N_1$ -- какое-либо подмножество множества $N$, то кардинальное число множества $N_1$ равно одному из предшествующих чисел $1, 2, 3, ..., \nu - 1$.
\end{theorem}

Первой доказывается теорема \ref{D}.
Доказательство производится методом от противного.
Далее доказывается по индукции теорема \ref{E}.
Теорема \ref{A} доказывается как следствие теоремы \ref{D} с применением метода математической индукции.
В доказательствах теорем \ref{B} и \ref{C} используется условие из определения \ref{def-less}.
Для доказательства теоремы \ref{B} также используется теорема \ref{A}, а для доказательства теоремы \ref{C} используется теорема \ref{E}.
После этого даётся ещё одна теорема:
\begin{theorem}
\label{F}
Если $K$ -- любое множество различных конечных кардинальных чисел, то среди них имеется число $\varkappa_1$, которое меньше остальных, а значит, является наименьшим из всех них.
\end{theorem}
Из неё следует последняя теорема этого параграфа.
\begin{theorem}
\label{G}
    Всякое множество $K = \{\varkappa\}$ различных кардинальных чисел можно представить в виде последовательности 
    $$K = \{\varkappa_1, \varkappa_2, \varkappa_3, ...\},$$ что $$\varkappa_1 < \varkappa_2 < \varkappa_3 < ...$$
\end{theorem}

\subsection{\S 6. Наименьшее трансфинитное кардинальное число алеф-нуль}
В этом параграфе Кантор вводит понятие трансфинитных кардинальных чисел и определяет $\aleph_0$.

По определению $\aleph_0 = \car{\{\nu\}}$, где $\{\nu\}$ -- множество всех конечных кардинальных чисел.
Для $\aleph_0$ показывается, что $\aleph_0 + 1 = \aleph_0$.
Из этого и теоремы \ref{A} следует, что $\aleph_0$ не является конечным кардинальным числом.
Также доказывается, что $\aleph_0$ больше любого конечного кардинального числа.
После этого доказывается, что $\aleph_0$ является наименьшей трансфинитной кардинальным числом.
Делается это при помощи двух теорем:
\begin{theorem}
    Всякое трансфинитное множество $T$ имеет подмножества с кардинальным числом $\aleph_0$.
\end{theorem}
\begin{theorem}
    Если $S$ -- трансфинитное множество с кардинальным числом $\aleph_0$ и $S_1$ -- какое-либо трансфинитное подмножество множества $S$, то $\car{S_1} = \aleph_0$.
\end{theorem}

Далее для $\aleph_0$ доказываются следующие свойства:
$$ \aleph_0 + \nu = \aleph_0, \nu \in \mathbb{N} $$
$$ \aleph_0 + \aleph_0 = \aleph_0 $$
$$ \aleph_0 \cdot \nu = \nu \cdot \aleph_0 = \aleph_0 $$
$$ \aleph_0 \cdot \aleph_0 = \aleph_0 $$
Для доказательства последнего явно строится биективное отображение из всех пар натуральных чисел $(\mu, \nu)$ во все натуральные числа $\lambda$:
$$ \lambda = \mu + (\mu + \nu - 1)(\mu + \nu - 2)/2. $$
Из последнего свойства вытекает следующее:
$$ \aleph_0^{\nu} = \aleph_0, \nu \in \mathbb{N} $$

Далее доказывается следующая теорема:
\begin{theorem}
    Всякое трансфинитное множество $T$ обладает тем свойством, что оно содержит подмножества $T_1$, которые эквивалентны ему.
\end{theorem}
Она противопоставлена следующей из теорем \ref{E} и \ref{A} теоремы
\begin{theorem}
    Всякое конечное множество обладает тем свойством, что оно не эквивалентно никакому своему подмножеству.
\end{theorem}

После этого Кантор говорит о том, что трансфинитные кардинальные числа образуют вполне упорядоченное множество,
и о том, что существуют законы, по которым можно строить \cit{ближайшее большее} для каждого кардинального числа и
для последовательности последовательных кардинальных чисел (например $\aleph_{\omega}$).
Кантор пишет, что для объяснения этих вещей он пользуется \cit{порядковыми типами,
теория которых излагается в нижеследующих параграфах.}

\subsection{\S 7. Порядковые типы просто упорядоченных множеств}
В седьмом параграфе Кантор вводит понятия порядкового типа, подобия множеств, класса типов и приводит несколько примеров, объясняющих суть этих понятий.

\cit{Просто упорядоченными} Кантор называет множества с введённым на них линейным порядком.
В качестве примера разных порядков на одном множестве он приводит 2 порядка на рациональных числах в интервале $(0, 1)$:
по величине и по сумме числителя и знаменателя в сокращённой форме (при равенстве -- по величине).
Определение типу он даёт следующее:
\begin{definition}
    Если на множестве $M$ задан порядок, то тип этого множества, обозначаемый $\overline{M}$, -- это 
    \cit{общее понятие, которое получается из $M$, когда мы отвлекаемся от качества элементов $m$, но сохраняем их порядковое расположение}.
\end{definition}
Подобие множеств определяется следующим образом:
\begin{definition}
    Упорядоченные множества $M$ и $N$ подобны, если между ними существует биекция, сохраняющая порядок. Обозначается $M \simeq N$
и о том, \end{definition}
Подобие рефлексивно, транзитивно, а ещё множества подобны тогда и только тогда, когда их порядковые типы совпадают.
Кантор пишет \cit{Если в порядковом типе $\overline{M}$ мы абстрагируемся и от расположения элементов,
то получим кардинальное число $\car{M}$ упорядоченного множества $M$, которое одновременно и является кардинальным числом порядкового типа $\overline{M}$.}
Из равенства типов следует равенство мощностей, обратное неверно.
Порядковые типы Кантор обозначает малыми греческими буквами.
Кардинальное число типа $\alpha$ он обозначает $\overline{\alpha}$.

В конечных множествах для каждого конечного кардинального числа существует только один соответствующий тип.
В трансфинитных множествах для одного кардинального числа существует несчётное число разных типов.
Совокупность этих типов образует класс типов, обозначающийся $[\mathfrak{a}]$.
Кантор собирается подробно описать $[\aleph_0]$.
Также он пишет, что если $[\mathfrak{a}]$ -- это множество, то его кардинальное число $\mathfrak{a'}$ отличается от $[\mathfrak{a}]$, и более того, оно больше.

Кантор вводит понятие обратного порядка.
\begin{definition}
    Если $M$ упорядоченное множество, то $^*M$ -- обратно упорядочено, если $\forall m_1, m_2 \in M~m_1 \leqslant m_2$ в порядке $M \iff m_1 \geqslant m_2$ в порядке $^*M$.
\end{definition}
Обратные порядки могут как совпадать с изначальными (например, это верно для конечных типов или обычного порядка рациональных чисел на $(0, 1)$,
который Кантор обозначает $\eta$), так и отличаться от них.
Тип может отображаться на себя многими способами (например, $\eta$), а может только единственным образом.
Единственным образом отображаются все конечные типы, а также Кантор говорит, что это будет верно для всех типов трансфинитных <<вполне упорядоченных множеств>>, которые образуют кардинальные числа.
Эти разные возможные количества автоморфизмов на себя Кантор иллюстрирует на примере типа натуральных чисел $\omega$ и типа целых чисел $^*\omega + \omega$.

В параграфе присутствует упоминание <<кратно упорядоченных множеств>>, без объяснения того, что это такое.
В конце параграфа Кантор говорит, что \cit{понятие <<порядкового типа>> охватывает <...> всё <<числоподобное>>, которое мыслимо вообще, и не допускает в этом смысле никакого дальнейшего обобщения.}
Он не даёт никакого описания того, что он считает <<числоподобным>>.
Далее он критикует статью Веронезе, у которого в определении равенства присутствует использование понятия равенства.

\subsection{\S 8. Сложение и умножение порядковых типов}
В восьмом параграфе Кантор вводит понятия сложения и умножения порядковых типов, их простые свойства и связь с операциями над кардинальными числами.

\begin{definition}
    Если $\alpha = \overline{M}, \beta = \overline{N}$, то $\alpha + \beta = \overline{(M, N)}$, где между элементами $M$ и $N$ порядок такой же и все элементы $M$ меньше элементов $N$.
\end{definition}
Для сложения верна ассоциативность, но не верна коммутативность, что показывается на примере $1 + \omega$ и $\omega + 1$.
\begin{definition}
    $\alpha \cdot \beta = \overline{S}$, где $S = \bigcup\limits_{n \in N}M_n$, где $M_n$ попарно не пересекаются и $\forall n \in N M_n \simeq M$.
    Порядок внутри $M_n$ сохраняется, элементы из разных $M_n$ сравниваются как соответствующие $n$.
\end{definition}
Для умножения верна ассоциативность и дистрибутивность относительно сложения, но только с одной стороны, а именно $$\alpha \cdot (\beta + \gamma) = \alpha \cdot \beta + \alpha \cdot \gamma.$$
Коммутативность не верна, что показывается на примере $2 \cdot \omega$ и $\omega \cdot 2$.

Кантор отмечает, что кардинальное число суммы типов равно сумме кардинальных чисел типов и кардинальное число произведения типов равно произведению кардинальных чисел типов.
Из этого следует, что равенства выражений, составленных из этих операций для порядковых типов, верны и для кардинальных чисел.

\subsection{\S 9. Порядковый тип $\eta$ множества всех рациональных чисел, которые больше 0 и меньше 1, в их естественном упорядочении}
В девятом параграфе даётся критерий того, что множество обладает порядковым типом $\eta$, и несколько следствий этого критерия.

В начале параграфа Кантор напоминает, что $\eta \in [\aleph_0]$, и делает он это при помощи
уже упомянутого упорядочивания множества $R = \mathbb{Q} \cap (0,1)$ с порядковым типом $\omega$.
Это упорядочивание обозначается $R_0 = {r_1, r_2, ...}$.
Далее отмечаются 2 свойства $\eta$: отсутствие наибольшего и наименьшего элемента и <<всюду плотность>>, то есть то, что между двумя разными элементами обязательно есть ещё один.
После этого формулируется следующая теорема:
\begin{theorem}
    Если просто упорядоченное множество $M$ удовлетворяет трём условиям:
    \begin{enumerate}
        \item $\car{M} = \aleph_0,$
        \item $M$ не имеет наинизшего и наивысшего по рангу элементов,
        \item $M$ всюду плотно,
    \end{enumerate}
    то порядковый тип у $M$ равен $\eta$, т.е. $$\overline{M} = \eta.$$
\end{theorem}
Для доказательства этой теоремы показывается, что если множество $M$ удовлетворяет этим свойствам, то $M \simeq R$.
Это, в свою очередь, делается построением биекции $R \rightarrow M$.
Образы элементов $R$ выбираются по очереди в соответствии с порядком $R_0$.
Элементу $r_\nu$ сопоставляется элемент $m_{i_\nu}$ такой, что он не был до этого выбран,
находится в том же отношении с уже выбранными $m_1, m_{i_2}, ..., m_{i_{\nu - 1}}$, что и $r_\nu$ с $r_1, r_2, ..., r_{\nu - 1}$ (такие есть в силу условий 2 и 3).
Среди всех таких подходящих элементов выбирается наименьший в каком-то заранее зафиксированном $\omega$-порядке $M$.

После этого для завершения доказательства показывается, что все элементы $M$ будут в какой-то момент выбраны.
Доказывается это при помощи \cit{полной индукции}.

В качестве следствия из этой теоремы приводится тот факт, что все типы $\eta + \eta$, $\eta\eta$, $(1 + \eta)\eta$, $(\eta + 1)\eta$, $(1 + \eta + 1)\eta$, $\eta \cdot \nu$, $\eta^{\nu}$, $^*\eta$ ($\nu$ конечно) равны $\eta$,
а типы $1 + \eta$,$ \eta + 1$,$ \nu \cdot \eta$,$ 1 + \eta + 1$,$ \eta$ все попарно различны при $\nu > 1$.
Также $\eta + 1 + \eta = \eta$, но $\eta + \nu + \eta \neq \eta$ при $\nu > 1$.

\subsection{\S 10. Фундаментальные последовательности, содержащиеся в трансфинитном упорядоченном множестве}
В десятом параграфе Кантор вводит понятие фундаментальной последовательности, предельного элемента последовательности,
главного элемента множества, плотного в себе множества, замкнутого множества.
Все эти понятия даются в смысле линейного порядка на множестве и с аналогичными понятиями для топологических или метрических пространств не совпадают.

Фундаментальной последовательностью в упорядоченном множестве Кантор называет подмножество, имеющие порядковый тип $\omega$ или $^*\omega$.
Первые называются возрастающими, вторые -- убывающими.
Этим определениям удовлетворяют обычные строго монотонные последовательности в числах.

Кантор вводит понятие перемежающихся последовательностей: две возрастающие (убывающие) фундаментальные последовательности перемежаются,
если правее (левее) каждого элемента одной последовательности есть элемент другой и наоборот,
а последовательности разной направленности: возрастающая $\{a_n\}$ и убывающая $\{b_n\}$ -- перемежаются,
если $\forall n, m \in \mathbb{N}~a_n > b_m$ и существует не более одного элемента $m_0$ такого, что $\forall n, m \in \mathbb{N}~b_n < m_0 < a_n$.

и о том, После этого утверждается, что перемежаемость транзитивна и что подпоследовательность, так же направленная, как и её содержащая последовательность,
всегда перемежается с содержащей её последовательностью.

Далее вводится понятие предельного элемента последовательности и главного элемента множества.
Я приведу определение для возрастающей последовательности, для убывающей оно аналогично.
\begin{definition}
    Если в $M$ существует элемент $m_0$, который по отношению к возрастающей фундаментальной последовательности $\{a_{\nu}\}$ занимает такое положение, что 
    \begin{enumerate}
        \item для всякого $\nu$ $$a_{\nu} < m_0$$
        \item для всякого элемента $m$ из $M$, который меньше $m_0$, существует такое определённое число $\nu_0$, что
            $$ a_{\nu} > m \text{ для } \nu \geqslant \nu_0,$$
            то мы будем называть $m_0$ <<предельным элементом последовательности $\{a_{\nu}\}$ в $M$>> и одновременно <<главным элементом множества $M$>>.
    \end{enumerate}
\end{definition}

После этого утверждается, что совпадение предельных элементов эквивалентно перемежаемости последовательностей (если хоть у одной есть предельный элемент).
Также утверждается, что подобие множеств сохраняет направленность, перемежение последовательностей и их предельные элементы, а значит, все эти объекты являются свойством порядкового типа.

После этого вводятся понятия плотного в себе множества (все элементы главные) и замкнутого множества (все фундаментальные последовательности имеют предельный элемент),
и сразу говорится, что эти свойства присущи порядковым типам.
Множества и типы, обладающие обоими этими свойствами, Кантор называет совершенными.

\section{Вывод}

Теория множеств и, вообще, понятие множества в настоящее время используется практически в любой области математики.
С наивной теорией множеств я познакомился ещё в 9 классе школы, она меня впечатлила своей логикой и необычностью её результатов.
Было очень интересно ознакомиться с тем, как её формулировал её создатель.
Местами названия и определения полностью совпадают с тем, что используется сейчас, местами нет.
Местами Кантор даёт определения через какие-то абстрактные рассуждения, а не строгие утверждения. 
Примером последнего можно, например, считать его определение мощности множества.
Также стоит отметить, что Кантор не рассматривает такие обычные операции над множествами, как пересечения, разность и объединение, когда множества имеют непустое пересечение.

С теорией же порядковых типов, которая тоже даётся здесь Кантором, я знаком не был.
Наиболее близкое к этому, что встречалось у меня в курсе, это обсуждение вполне упорядоченных множеств и теорий упорядоченных множеств на лекциях по математической логике.
Было интересно познакомиться с новой для меня областью математики, особенно в изложении одного из её создателей.

Так же было очень интересно узнать и биографию человека, сделавшего так много для математики, поскольку современная математика без теории множеств не могла бы существовать. 





\newpage
\bibliographystyle{ugost2008}
\begin{thebibliography}{}
    \bibitem{orig} Cantor G. Beiträge zur Begründung der transfiniten Mengenlehre //Mathematische Annalen. -- 1897. -- Т. 49. -- №. 2. -- С. 207-246.
    \bibitem{used} Кантор Г. Труды по теории множеств. М., Наука. 1985 С. 173-200
    \bibitem{bio} Синкевич Г. И. Георг Кантор из Санкт-Петербурга:(с комментариями Георга Зингера): монография. 2-е изд., перераб. и доп //Санкт-Петербург: Санкт-Петербургский государственный архитектурно-строительный университет. -- 2023.
\end{thebibliography}
\end{document}
