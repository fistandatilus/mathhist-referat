% !TeX spellcheck = ru_RU
\documentclass[a4paper,12pt]{extarticle}
\usepackage[T2A]{fontenc}
\usepackage[utf8]{inputenc}
\usepackage[english,russian]{babel}
\usepackage{amsmath}
\usepackage{amssymb}
\usepackage{amsthm}
\usepackage{listings}
\usepackage{hyperref}

\usepackage{xcolor}

\usepackage{indentfirst}
\usepackage[left=25mm, top=35mm, right=25mm, bottom=25mm, nohead, footskip=5mm]{geometry}
\renewcommand{\baselinestretch}{1.5}
\usepackage[indent=1.25cm]{parskip}

\newtheorem{theorem}{Теорема}[subsection]
\renewcommand{\thetheorem}{\arabic{subsection}.\arabic{theorem}}
\newtheorem{utv}{Утверждение}
\theoremstyle{definition}
\newtheorem{definition}{Определение}
\newcommand{\cit}[1]{<<\textit{#1}>>}
\newcommand{\car}[1]{\overline{\overline{#1}}}

\author{Борисенков Никита}



\begin{document}

\begin{titlepage}
    \newpage
    \begin{center}
        Московский государственный университет имени M.В. Ломоносова
        Механико-математический факультет\\
    \end{center}

    \vspace{8em}

    \begin{center}
        \Large Реферат\\
    \end{center}

    \vspace{2em}

    \begin{center}
        \textsc{\textbf{Г. Кантор. К обоснованию учения о трансфинитных множествах}}
    \end{center}

    \vspace{20em}



    \newbox{\lbox}
    \newlength{\maxl}
    \setlength{\maxl}{\wd\lbox}
    \hfill\parbox{13cm}{
        \hspace*{5cm}\hspace*{-5cm}Студент: \qquad\qquad\hbox to \maxl{Борисенков Никита Николаевич\hfill}\\
        \\
        \hspace*{5cm}\hspace*{-5cm}Группа:\qquad\qquad $\;\:$ \hbox to\maxl{410}\\
    }


    \vspace{\fill}

    \begin{center}
        Москва \\2025
    \end{center}

\end{titlepage}
\newpage

\section{Основная часть}
\subsection{\S 1. Понятие мощности или кардинального числа}
В первом параграфе Кантор вводит понятие множества, объединение множеств и базовые определения и факты о кардинальности.
Он не даёт чёткого определения множества, но говорит, что это \cit{
соединение в некое целое $M$ определённых хорошо различимых предметов $m$ нашего созерцания или нашего мышления.
}
Объединение непересекающихся множеств, которые мы сейчас обозначаем как $M \sqcup N$, Кантор обозначает так: $(M, N)$.
Обозначения для подмножества он не даёт, но даёт следующее простое утверждение о них:
\cit{Если $M_2$ является частью $M_1$, а $M_1$ -- частью $M$, то и $M_2$ является часть $M$.
}

Далее Кантор вводит понятие мощности или иначе кардинального числа, которое он определяет в том же стиле, в каком он определял множества.
Конкретно его определение выглядит так:
\cit{<<Мощностью>> или <<кардинальным числом>> множества $М$ мы называем то общее понятие,
которое получается при помощи нашей активной мыслительной способности из $M$,
когда мы абстрагируемся от качества его различных элементов $m$ и от порядка их задания.}
Это определение можно понять как то, что мощность -- это особое множество единиц.
В дальнейшем он использует её, в том числе, и как множество
Мощность множества он обозначает так: $\car{M}$

Сразу после определения мощности Кантор определяет эквивалентность множеств, и в этот раз определение совпадает с современным.
\begin{definition}
Множества $M$ и $N$ называются эквивалентными, если между ними существует взаимно-однозначное соответствие. Обозначается так: $M \sim N$ или $N \sim M$.
\end{definition}

Далее Кантор замечает, что эквивалентность множеств может задаваться отображением, которое будет заданный элемент одного множества переводить в заданный элемент второго множества.
После этого он утверждает, что для отношения эквивалентности множеств верны симметричность и транзитивность.
Далее Кантор формулирует и доказывает следующее утверждение:
\begin{utv}
\label{simeq}
    $M \sim N \iff \car{M} = \car{N}$.
\end{utv}
Завершается параграф следующим утверждением:
\begin{utv}
\label{manysim}
Если множества $M_i$ попарно не пересекаются, множества $M_i'$ тоже попарно не пересекаются и $M_i \sim M_i'$, то тогда $\bigcup\limits_{i \in I} M_i \sim \bigcup\limits_{i \in I} M_i'$
\end{utv}

\subsection{\S 2. <<Больше>> и <<меньше>> для мощностей}
Во втором параграфе Кантор вводит отношения больше и меньше на кардинальных числах.
\begin{definition}
    \label{def-less}
    Кардинальные числа $\mathfrak{a} = \car{M}, \mathfrak{b} = \car{N}$ состоят в отношении $\mathfrak{a} < \mathfrak{b}$, если верны 2 условия:
    \begin{enumerate}
        \item $\forall M_1 \subset M, M_1 \nsim N$
        \item $\exists N_1 \subset N, N_1 \sim M$
    \end{enumerate}
\end{definition}
Далее замечается, что из этого определения следует, что если верно $\mathfrak{a} < \mathfrak{b}$, то не верно ни $\mathfrak{a} = \mathfrak{b}$, ни $\mathfrak{b} < \mathfrak{a}$,
а так же то, что для него верна транзитивность.
Отмечаются следующие соотношения про подмножества:
$$
P_1 \subset P, \mathfrak{a} < \car{P_1} \Rightarrow \mathfrak{a} < \car{P}
$$
$$
P_1 \subset P, \car{P} < \mathfrak{b} \Rightarrow \car{P_1} < \mathfrak{b}
$$

При этом Кантор указывает, что в данный момент рассуждений невозможно показать, что любые 2 кардинальных числа состоят в отношении <, > или =,
и приводит несколько утверждений, которые из этого следуют и будут обсуждены позднее.

\subsection{\S 3. Сложение и умножение мощностей}
В третьем параграфе Кантор вводит сложение и умножение мощностей, а так же операции над множествами, которые к ним приводят.
Объединение множеств, которое он ввёл в первом параграфе он называет суммой.
Сумму же кардинальных чисел Кантор вводит следующим образом: $\mathfrak{a} = \car{M}, \mathfrak{b} = \car{N} \Rightarrow \mathfrak{a + b} = \car{M \sqcup N}$.
Корректность этого определения следует из утверждений \ref{simeq} и \ref{manysim}.
Из того, что мощность не зависит от порядка элементов получаются коммутативность и ассоциативность сложения.

После этого вводится умножение множеств, которое сейчас известно как декартово произведение.
Кантор даёт следующее определение: \cit{Каждый элемент $m$ множества M можно объединить с каждым элементом $n$ множества N в новый элемент $(m, n)$;
для множества всех этих объединений $(m, n)$ мы примем обозначение $(M \cdot N)$. Назовём его <<\textnormal{Произведение множеств $M$ и $N$}>>.}
Далее показывается, что если $M \sim M'$ и $N \sim N'$, то $(M \cdot N) \sim (M' \cdot N')$.
Из этого получается, что можно корректно определить умножение кардинальных чисел: $\mathfrak{a \cdot b} = \car{(M \cdot N)}$, если $\mathfrak{a} = \car{M}, \mathfrak{b} = \car{N}$.
Так же Кантор показывает, что для умножения верна коммутативность, ассоциативность и есть дистрибутивность относительно сложения.

\subsection{\S 4. Возведение мощностей в степень}
В четвёртом параграфе определяется возведение мощностей в степень.
Кантор делает это через множество всех отображений $M \rightarrow N$.
Он называет такие отображения \cit{покрытие $N$ посредством $M$}, а всё множество обозначает $(N|M)$
Возведение в степень кардинальных чисел определяется так: $\mathfrak{a^b} = \car{(N|M)}$.
Далее сообщается, что для возведения в степень верны следующие соотношения:
$$\mathfrak{a^b \cdot a^c = a^{b+c}}$$
$$\mathfrak{a^c \cdot b^c = (a \cdot b)^c}$$
$$\mathfrak{(a^b)^c = a^{b \cdot c}}$$

В примечании к этому параграфу отмечена важность этих формул и приведён пример их использования для доказательства того,
что мощность континуума в счётной степени совпадает с мощностью континуума.

\subsection{\S 5. Конечные кардинальные числа}
В пятом параграфе рассматривается применение общего учения о кардинальных числах к конечным множествам.
Сначала Кантор строит натуральные числа на основе мощностей конечных множеств.
Множеству из одного элемента $E_0 = \{e_0\}$ сопоставляется единица $1 = \car{E_0}$, далее $E_1 = E_0 \sqcup {e_1}, \car{E_1} = 2$ и так далее.
В общем случае $\nu = \car{E_{\nu - 1}}, E_{\nu} = E_{\nu} \sqcup \{e_{\nu}\} = \{e_0, ..., e_{\nu}\}$.
Из определения суммы мощностей следует $\car{E_{\nu}} = \car{E_{\nu-1}} + 1$.

Далее Кантор формулирует 3 теоремы о свойствах только что определённых чисел.
Текст теорем перенесён из самой работы.
\begin{theorem}
\label{A}
Все члены неограниченной последовательности конечных кардинальных чисел $$1, 2, 3, ..., \nu,...$$ отличны друг от друга.
\end{theorem}
\begin{theorem}
\label{B}
Каждое из этих чисел $\nu$ больше, чем предшествующее ему и меньше, чем следующее за ним.
\end{theorem}
\begin{theorem}
\label{C}
Не существует кардинального числа, которое по своей величине было бы расположено между двумя соседними числами $\nu$ и $\nu + 1$
\end{theorem}
Доказательства этих теорем основываются на двух других теоремах
\begin{theorem}
\label{D}
Если $M$ -- множество, обладающее свойством не иметь одинаковой мощности ни с каким своим подмножеством, то и множество $(M, e)$,
получаемое из $M$ присоединением  одного нового элемента $e$, обладает тем же свойством не иметь одинаковой мощности ни с каким из его подмножеств.
\end{theorem}
\begin{theorem}
\label{E}
Если $N$ -- множество с конечным кардинальным числом $\nu$, а $N_1$ -- какое-либо подмножество множества $N$, то кардинальное число множества $N_1$ равно одному из предшествующих чисел $1, 2, 3, ..., \nu - 1$.
\end{theorem}

Первой доказывается теорема \ref{D}.
Доказательство производится методом от противного.
Далее доказывается по индукции теорема \ref{E}.
Теорема \ref{A} доказывается как следствие теоремы \ref{D} с применением методом математической индукции.
В доказательствах теорем \ref{B} и \ref{C} используется условие из определения \ref{def-less}.
Для доказательства теоремы \ref{B} так же используется теорема \ref{A}, а для доказательства теоремы \ref{C} используется теорема \ref{E}.
После этого даётся ещё одна теорема:
\begin{theorem}
\label{F}
Если $K$ -- любое множество различных конечных кардинальных чисел, то среди них имеется число $\varkappa$, которое меньше остальных, а значит является наименьшим из всех них.
\end{theorem}
Из неё следует последняя теорема этого параграфа.
\begin{theorem}
\label{G}
    Всякое множество $K = \{\varkappa\}$ различных кардинальных чисел можно представить в виде последовательности 
    $$K = \{\varkappa_1, \varkappa_2, \varkappa_3, ...\},$$ что $$\varkappa_1 < \varkappa_2 < \varkappa_3 < ...$$
\end{theorem}

\subsection{\S 6. Наименьшее трансфинитное кардинальное число алеф-нуль}
В этом параграфе Кантор вводит понятие трансфинитных кардинальных чисел и определяет $\aleph_0$.
По определению $\aleph_0 = \car{\{\nu\}}$, где $\{\nu\}$ -- множество всех конечных кардинальных чисел.
Для $\aleph_0$ показывается, что $\aleph_0 + 1 = \aleph_0$.
Из этого и \ref{A} следует, что $\aleph_0$ не является конечным кардинальным числом.
Так же доказывается, что $\aleph_0$ больше любого конечного кардинального числа.
После этого доказывается, что $\aleph_0$ является наименьшей трансфинитной кардинальным числом.
Делается это при помощи двух теорем:
\begin{theorem}
    Всякое трансфинитное множество $T$ имеет подмножества с кардинальным числом $\aleph_0$.
\end{theorem}
\begin{theorem}
    Если $S$ -- трансфинитное множество с кардинальным числом $\aleph_0$ и $S_1$ -- какое-либо трансфинитное подмножество множества $S$, то $\car{S_1} = \aleph_0$.
\end{theorem}

Далее для $\aleph_0$ доказываются следующие свойства:
$$ \aleph_0 + \nu = \aleph_0, \nu \in \mathbb{N} $$
$$ \aleph_0 + \aleph_0 = \aleph_0 $$
$$ \aleph_0 \cdot \nu = \nu \cdot \aleph_0 = \aleph_0 $$
$$ \aleph_0 \cdot \aleph_0 = \aleph_0 $$
Для доказательства последнего явно строится биективное отображение из всех пар натуральных чисел $(\mu, \nu)$ во все натуральные числа $\lambda$:
$$ \lambda = \mu + (\mu + \nu - 1)(\mu + \nu - 2)/2. $$
Из последнего свойства вытекает следующее:
$$ \aleph_0^{\nu} = \aleph_0, \nu \in \mathbb{N} $$

Далее доказывается следующая теорема:
\begin{theorem}
    Всякое трансфинитное множество $T$ обладает тем свойством, что оно содержит подмножества $T_1$, которые эквивалентны ему.
\end{theorem}
Которая противопоставлена следующей из теорем \ref{E} и \ref{A} теоремы
\begin{theorem}
    Всякое конечное множество обладает тем свойством, что оно не эквивалентно никакому своему подмножеству.
\end{theorem}

После этого Кантор говорит о том, что трансфинитные кардинальные числа образуют вполне упорядоченное множество,
что существуют законы, по которым можно строить \cit{ближайшее большее} для каждого кардинального числа и
для последовательности последовательных кардинальных чисел (например $\aleph_{\omega}$).
Кантор пишет, что для объяснения этих вещей он пользуется \cit{порядковыми типами,
теория которых излагается в нижеследующих параграфах.}

\subsection{\S 7. Порядковые типы просто упорядоченных множеств}
В седьмом параграфе Кантор вводит понятия порядкового типа, подобия множеств, класса типов и приводит несколько примеров, объясняющих суть этих понятий.
\cit{Просто упорядоченными} Кантор называет множества с введённым на них линейным порядком.
В качестве примера разных порядков на одном множестве он приводит 2 порядка на рациональных числах в интервале $(0, 1)$:
по величине и по сумме числителя и знаменателя в сокращённой форме (при равенстве -- по величине).
Определение типу он даёт следующее:
\begin{definition}
    Если на множестве $М$ задан порядок, то тип этого множества, обозначаемый $\overline{M}$, -- это 
    \cit{обще понятие, которое получается из $M$, когда мы отвлекаемся от качества элементов $m$, но сохраняем их порядковое расположение}.
\end{definition}
Подобие множеств определяется следующим образом:
\begin{definition}
    Упорядоченные множества $M$ и $N$ подобны, если между ними существует биекция, сохраняющая порядок. Обозначается $M \simeq N$
\end{definition}
Подобие рефлексивно, транзитивно, а ещё множества подобны тогда и только тогда, когда их порядковые типы совпадают.
Кантор пишет \cit{Если в порядковом типе $\overline{M}$ мы абстрагируемся и от расположения элементов,
то получим кардинальное число $\car{M}$ упорядоченного множества $M$, которое одновременно и является кардинальным числом порядкового типа $\overline{M}$.}
Из равенства типов следует равенство мощностей, обратное неверно.
Порядковые типы Кантор обозначает малыми греческими буквами.
Кардинальное число типа $\alpha$ он обозначает $\overline{\alpha}$.

В конечных множествах для каждого конечного кардинального числа существует только один соответствующий тип.
В трансфинитных множествах для одного кардинального числа существует несчётное число разных типов.
Совокупность этих типов образует класс типов, обозначающийся $[\mathfrak{a}]$
Кантор собирается подробно описать $[\aleph_0]$.
Так же он пишет, что если $[\mathfrak{a}]$ это множество, то его кардинальное число $\mathfrak{a'}$ отличается от $[\mathfrak{a}]$, и более того, оно больше.

Кантор вводит понятие обратного порядка.
\begin{definition}
    Если $M$ упорядоченное множество, то $^*M$ -- обратно упорядочено, если $\forall m_1, m_2 \in M m_1 \leqslant m_2$ в порядке $M, \iff m_1 \geqslant m_2$ в порядке $^*M$.
\end{definition}
Обратные порядки могут как совпадать с изначальными (например это верно для конечных типов или обычного порядка рациональных чисел на $(0, 1)$,
который Кантор обозначает $\eta$), так и отличаться от них.
Тип может отображаться на себя многими способами (например $\eta$), а может только единственным образом.
Единственным образом отображаются все конечные типы, а так же Кантор говорит, что это будет верно про все типы трансфинитных <<вполне упорядоченных множеств>>, которые образуют кардинальные числа.
Эти разные возможные количества автоморфизмов на себя кантор иллюстрирует на примере типа натуральных чисел $\omega$ и типа целых чисел $^*\omega + \omega$.

В параграфе присутствует упоминание <<кратно упорядоченных множеств>>, без объяснения того, что это такое.
В конце параграфа Кантор говорит, что \cit{понятие <<порядкового типа>> охватывает <...> всё <<числоподобное>>, которое мыслимо вообще, и не допускает в этом смысле никакого дальнейшего обобщения.}
Он не даёт никакого описания того, что он считает <<числоподобным>>.
Далее он критикует статью Веронезе, у которого в определении равенства присутствует использование понятия равенства.

\subsection{\S 8. Сложение и умножение порядковых типов}
В этом параграфе Кантор вводит понятия сложения и умножения порядковых типов, их простые свойства и связь с операциями над кардинальными числами.
\begin{definition}
    Если $\alpha = \overline{M}, \beta = \overline{N}$, то $\alpha + \beta = \overline{(M, N)}$, где между элементами $M$ и $N$ порядок такой же и все элементы $M$ меньше элементов $N$.
\end{definition}
Для сложения верна ассоциативность, но не верна коммутативность, что показывается на примере $1 + \omega$ и $\omega + 1$.
\begin{definition}
    $\alpha \cdot \beta = \overline{S}$, где $S = \bigcup\limits_{n \in N}M_n$, где $M_n$ попарно не пересекаются и $\forall n \in N M_n \simeq M$.
    Порядок внутри $M_n$ сохраняется, элементы из разных $M_n$ сравниваются как соответствующие $n$.
\end{definition}
Для умножения верна ассоциативность и дистрибутивность относительно сложения, но только с одной стороны, а именно $$\alpha \cdot (\beta + \gamma) = \alpha \cdot \beta + \alpha \cdot \gamma.$$
Коммутативность не верна, что показывается на примере $2 \cdot \omega$ и $\omega \cdot 2$.

Кантор отмечает, что кардинальное число суммы типов равно сумме кардинальных чисел типов и кардинальное число произведения типов равно произведению кардинальных чисел типов.
Из этого следует, что равенства выражений, составленных из этих операций для порядковых типов верны и для кардинальных чисел.






\end{document}
